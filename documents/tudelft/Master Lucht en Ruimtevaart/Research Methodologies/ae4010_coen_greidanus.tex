\documentclass{article}

\usepackage{hyperref}

\title{Numerical convection schemes}
\author{Coen Greidanus, 1285688\\
Aerodynamics}

\begin{document}
\maketitle

\begin{abstract}

Say very briefly 1) what the project is about, 2) what the main content will be, 3) what the main aim or objectives are, and 4) what the main findings could be. End with a strong sentence that highlights the significance of the work to be undertaken and any long-standing contribution to the body of knowledge. Remember that this is a proposal of work to be done and so you might also say something about the motivation and feasibility.

\end{abstract}

\section{Introduction}
\label{sec:intro}

Introduce the general area of interest of the project, setting out any advancements and challenges of interest. What is the relevance and context of the work at an academic and applied/industry level? Then introduce more fully the specific investigation addressed in the project proposal and perhaps even set out the main goal of the work (Note: different to the research question!). Say very briefly what is then to come in the layout of the proposal. Note: the introduction should include general references to back up the points made.

\section{State of the art / Literature review}
\label{sec:litreview}

This is a detailed part of the proposal that rigorously reviews what work has been already carried out by other academics in this area while also benchmarking industry best practice. The researcher is trying to establish: 1) what research areas are relevant, and 2) what the current understanding is along with any opposing views. It is very nice to end with some sort of synthesis of the presented State-of-the-art to link explicitly to the work in the proposal, especially with regards to the following Section. This might even include a statement of what the author sees their work adding to the body of knowledge.

Please note that the literature review deliverable is not an extended version of this section of the project proposal.

\section{Research question, aims and objectives}
\label{sec:objective}

What is the main research question to be solved in reaching the project goal? There can be more than one but be focused. Split the research question into sub-questions, with each sub-question having further questions. The lower level questions will, together, provide the answer for the higher level questions. These research questions should be very precise and like a requirement, be unambiguous, unique, measurable, and answerable in a meaningful way.

The objective then is basically the project goal, again clearly stated in terms of what the researcher wants to achieve, and by which means you will achieve this. This is then followed by tangible sub-goals that will be necessary to make this happen. These sub goals can then be developed into task blocks in the project plan/Gantt chart.

Make the novelty and innovation clear!

Again, remember that this is a proposal of work to be done and so you might also say something about the motivation and feasibility.


\section{Theoretical Content/Methodology}
\label{sec:theory}

What is the theoretical basis of the work to be undertaken? Is there a hypothesis to be tested? Are a couple of theoretical approaches going to be used together in a hybrid approach. What are the steps to be undertaken in the project � linking to the objectives established in Section~\ref{sec:objective}.

\section{Experimental Set-up}
\label{sec:experiment}

What is your laboratory set-up or in the field set-up, presented so readers can better informed and critical of any limitations etc of your research environment and set-up, etc. For example, how will the collaboration with industry work or otherwise what are any practical implications of your Methodology from Section 3? Experiments are more then just questionnaires, interviews or physical tests in a laboratory. Programming and computer modeling are also considered experiments so their set up and limitations can also be discussed here.

\section{Results, Outcome and Relevance}
\label{sec:result}

What data etc will you be working with, which variables and parameters, and what type of results do you want to investigate? Then go on to try and project the sort of outcome you are interested in and of course ultimately what the relevance of that is.

\section{Project Planning and Gantt Chart}
\label{sec:planning}

Look at the logistics of carrying out the work, develop the intended work into work packages (from the tasks mentioned in Section 3) and then incorporate all that into a schedule of work through a Gantt chart (or Integrated Master Plan/Integrated Master Schedule). You must, at a reasonably high level, try to estimate time required and any resource constraints etc and then fit that into a schedule. A Work Breakdown Structure and Work Flow Diagram can be very helpful (ref: DSE). You will put in three key review points: 1) the Kick-off Review/prelim meeting; 2) the Mid-term Review; and 3) the Green Light Review.
Remember holidays etc, preparing deliverables etc and that some tasks will be concurrent, running in parallel. Finally, the Gantt chart should visibly show any key review points, milestones (points of specific achievement relative to the end goal) and deliverables (tangible outputs such as reports, presentations, papers, manuals, tools, workshops, etc).

\section{Conclusions}
\label{sec:conclusions}

The conclusions regarding what you are proposing should be written in a precise, unique, clear and accurate manner. Always check if they are well supported by the work you presented in the project proposal and check them against the main literature so that you can make a statement about the longer-term impact of your work on the body of knowledge. Lift the most important conclusions into the Executive Summary and check that both are consistent, also with the Introduction. This is done because the Executive Summary, Introduction and Conclusion form the key points of entry and exit into the work and make a big impact on accessibility and getting across the relevance!

\bibliographystyle{plain}
\bibliography{ae4010_project_template}

List all references consistently, using one of the preferred approaches, for example the Harvard referencing system. A useful link on how to quote and reference properly can be found in the Guide to Harvard referencing from the Anglia Ruskin University Library in the UK: http://libweb.anglia.ac.uk/referencing/harvard.htm.

The key thing is that the referred to author is given credit through their earlier work, that this is dated to show the chronological order of developments, and that the reader has enough information to go and find that specific reference. Relative to the latter point: where was the conference or if a journal review what was the volume or edition number and certainly page numbers.

The strongest references are ones that have been reviewed prior to publication (journals for example) and the weakest are web sites and popular publications. Only reputable websites (from a society or major industry player) should be included and the date of access should be noted. Preferably stay away from web references as they are so uncontrolled as sources of information.

Note: web references, technical reports, standard textbooks etc. do not count towards the number of quality references used. The number of quality references should always greatly outnumber the number of general references. It looks unprofessional to refer to the textbooks and templates of this course.


\LaTeX {} template based on \cite{template}.

\end{document}
